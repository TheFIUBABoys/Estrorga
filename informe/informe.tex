\documentclass[a4paper,10pt]{article}

\usepackage{graphicx}
\usepackage[utf8]{inputenc}
\usepackage[spanish]{babel}
\usepackage{hyperref}
\usepackage{listings}
\usepackage{verbatim}
\usepackage[top=2cm, bottom=1.5cm, left=2.5cm, right=1cm]{geometry}
\usepackage{pdfpages}
\usepackage{enumitem}

\title{		\textbf{Título}}

\author{	Lucas Simonelli, \textit{Padrón Nro. 93111}                     \\
            \texttt{ lucasp.simonelli@gmail.com }                                              \\[2.5ex]
            Tomás Boccardo, \textit{Padrón Nro. 93637}                     \\
            \texttt{ tomasboccardo@gmail.com}                                              \\[2.5ex]
            Andrés Sanabria, \textit{Padrón Nro. 93403}                     \\
            \texttt{ andresg.sanabria@gmail.com  }                                              \\[2.5ex]
             Agregate master, \textit{Padrón Nro. 9XXXX}                     \\
            \texttt{a@a.com  }                                              \\[2.5ex]
            \normalsize{2do. Cuatrimestre de 2013}                                      \\
            \normalsize{71.12, Estructura de las organizaciones}  \\
            \normalsize{Facultad de Ingeniería, Universidad de Buenos Aires}            \\
       }
\date{}
\begin{document}





\maketitle
\thispagestyle{empty}   % quita el nómero en la primer pógina



\begin{abstract}
Acá va un resumen del trabajo práctico
\end{abstract}

\newpage
\tableofcontents
\newpage
\section{Introducción}
Intro
\section{Empresa relevada}
	\subsection{Preguntas}
		\subsubsection{Preguntas introductorias}
			\begin{enumerate}
				\item \textit{¿Cuál es la razón social de la empresa?}\\
				Tecnología Contra Incendios S.A.
				
				\item \textit{¿En qué lugar se encuentra localizada la empresa?}\\
				La empresa está ubicada en Carapachai, Munro. La dirección es Gobernador Emilio Castro 3365.			
				
				\item \textit{Mencione los acontecimientos más destacados en la evolución de la empresa.}\\
				La empresa nace en el año 1965, bajo el nombre de \textbf{TECIN ARGENTINA S.R.L.} Al poco tiempo se transforma en una S.A. y obtiene la representación y distribución, de empresas internacionales de reconocida trayectoria en su especialidad como Rosenbauer, Angus, Reliable, Walter Kidde, Total y Cerberus. En 1982 la empresa construye la primer autobomba argentina tras asociarse con la empresa austriaca Rosenbauer K.G. (hoy Rosenbauer International A.G.).\\
				En la década del 90, sus accionistas deciden formar sociedades independientes para atender unidades de negocios y mercados diferentes, siendo una de ellas: 
				\begin{itemize}
					\item \textbf{Tecnología Contra Incendios S.A.}, es hoy 100\% propiedad de accionistas argentinos y esta dedicada a la fabricación de vehículos y equipos contra incendios y rescate.
				\end{itemize}
				\item \textit{¿Cuál es el rubro al que se dedica la empresa?}\\
				Equipos de seguridad contra incendios (vehículos/equipos/herramientas).
			
				\item \textit{¿Cuál es la línea de productos ofrecida por la organización?}\\	
				La línea de productos ofrecida es de:\\
				\textbf{Vehículos}:
				\begin{itemize}
					\item Contra incendios urbanos, industriales y forestales.
					\item De rescate y manejo de sustancias peligrosas.
				\end{itemize}
				\textbf{Productos/equipos}:
				\begin{itemize}
					\item Motobombas portátiles.
					\item De protección personal y respiratoria.
					\item Para controlar incendios: mangueras, matafuegos, etc.
				\end{itemize}	
												
				\item \textit{¿La empresa cumple algún tipo de certificación?}\\
				Sí, la empresa está certificada bajo las normas IRAM ISO 9001:2008
						
			\end{enumerate}
			
			
		\subsubsection{Comercial}
		
		
			\begin{enumerate}[resume]
			
				\item \textit{¿Cuáles son los productos más vendidos por la empresa?}\\
			
				Respuesta
				
				\item \textit{¿Qué productos son exportados por la empresa?}\\
				
				Respuesta
				
				\item \textit{¿Cuáles son los principales destinos de las exportaciones?}\\
				
				Respuesta
				
				\item \textit{¿Qué porcentaje del mercado concentran los productos fabricados?}\\
				
				Respuesta
				
				\item \textit{¿Qué métodos de publicidad utiliza la empresa?}\\
				
				Respuesta
				
				\item \textit{¿Cuáles son los principales clientes de la empresa?}\\
				
				Respuesta
				
				\item \textit{¿Qué métodos de realización de pedidos están disponibles?}\\
				
				Respuesta
				
				\item \textit{¿Cómo se planifica la distribución de los pedidos?}\\
				
				Respuesta
				
			\end{enumerate}
			
\subsubsection{Producción}
			\begin{enumerate}[resume]
				\item \textit{¿En qué consiste el proceso productivo del producto principal?}\\
				
				Respuesta
				
				
				\item \textit{¿Cuáles son las principales materias primas?}\\
				
				Respuesta
				
				\item \textit{¿Cuáles son los proveedores más importantes de la empresa?}\\
				
				Respuesta
				
				\item \textit{¿Produce algún insumo necesario para la manufactura del producto final?}\\
				
				Respuesta
				
				\item \textit{¿Poseen un equipo que se encargue del mantenimiento, o es un servicio tercerizado?}\\
				
				Respuesta
				
				\item \textit{¿Necesita operarios calificados? ¿Realiza algún tipo de capacitación?}\\
				
				Respuesta
				
				\item \textit{¿Qué insumos son importados por la organización?}\\
				
				Respuesta				
			
			\end{enumerate}			
			
	\subsubsection{R.R. H.H.}
		
		
			\begin{enumerate}[resume]

			\item \textit{¿Cuántos empleados tiene la organización?}\\
			La empresa tiene 50 empleados.
				
			\item \textit{¿Qué beneficios posee el personal de la empresa?}\\
			
			Respuesta
			
			\item \textit{¿Cuáles son los principales sectores de la empresa?}\\
			
			Respuesta
			
			\item \textit{¿Tiene la empresa un organigrama propio?}\\
			Si, la empresa posee organigrama propio.
					
			\item \textit{¿Puede la empresa proporcionar dicho organigrama?}\\
			Si, la empresa puede proporcionar dicho organigrama
			
			\item \textit{¿Cómo es el proceso de selección del personal? }\\
			
			Respuesta
			
			\item \textit{¿Se realizan evaluaciones periódicas del desempeño del personal?}\\
			
			Respuesta
			
			\item \textit{¿Hay bonificaciones salariales por buen desempeño?}\\
			
			Respuesta
			
			\item \textit{¿Se realizan capacitaciones periódicas a los empleados?}\\
			
			Respuesta
			
			\item \textit{¿Se organizan eventos para fomentar la relación interpersonal entre los empleados?}\\
			
			Respuesta
			
			\item \textit{¿Contratan servicios tercerizados, como por ejemplo: estudio contable, jurídico, controles de calidad, etc.?}\\

			\end{enumerate}
			
			
		\subsubsection{Finanzas}
		
		
			\begin{enumerate}[resume]
			
			\item \textit{¿Cuáles son los plazos promedio para la cobranza de las facturas?}\\
			
			Respuesta
			
			\item \textit{¿Recibe algún beneficio impositivo por parte del estado?}\\

			Respuesta
			
			\end{enumerate}
		
		\subsubsection{Control de Calidad}
		
		
			\begin{enumerate}[resume]

			\item \textit{¿Cuáles es la política de calidad de la organización?}\\
			La política de calidad de la empresa presenta los siguientes objetivos:
			\begin{itemize}
				\item El cumplir con los compromisos contraídos con los clientes y superar sus expectativas, constituyen una obligación para todo el personal.
				\item Asumir como  indispensable el cumplimiento  de los requerimientos del Sistema de 
Gestión  de la Calidad y comprometerse a su continua mejora.
				\item Nos comprometerse a evaluar, motivar y capacitar a los recursos humanos, en forma 
permanente. 
				\item Mantener un contacto productivo con los Proveedores, para mejorar las prestaciones 
y productos ofrecidos. 
				\item Difundir al personal los objetivos comprometidos en la presente política. 
			\end{itemize}
	
			\item \textit{¿Cómo realizan los controles de calidad sobre la producción?}\\
			
			Respuesta
			
			\item \textit{¿Realizan auditorías internas? ¿Con qué frecuencia?}\\

			Respuesta
			
			\item \textit{¿Realizan auditorías externas? ¿Con qué frecuencia?}\\
			Respuesta
			
			\end{enumerate}	
			
\section{Casos de estudio}
	\subsection{Caso 1: Elevadores Hércules S.A.}
		\subsubsection{Enunciado}
		Elevadores Hércules S.A., establecida en Buenos Aires en 1919 como una oficina de
contratistas, se desarrolló al punto de transformarse en una de las compañías más
importantes del mundo. En 1966, la compañía producía 1650 elevadores y en 1974
llegó a 7.850 unidades, inclusive escaleras mecánicas. Aunque su planta principal está
ubicada en Buenos Aires, tiene oficinas comerciales en las 18 ciudades más importantes
del país participando con más del 60% del mercado nacional. A partir de 1970 el número
de edificios comenzó a aumentar considerablemente. Los pedidos de los clientes tendían
a alcanzar límites que sobrepasaban la capacidad de producción de la fábrica. Los
atrasos en la entrega de pedidos llegaron al punto de provocar serios conflictos entre los
departamentos de ventas y producción.\\
En función de lo anterior, la alta dirección de la compañía decidió perfeccionar el
sistema de planeamiento y control de la fábrica.\\ \\
\textbf{Principales características del sistema de producción}\\
La producción de elevadores requiere cerca de 6.000 diferentes grupos de piezas de
varios tipos o medidas y aproximadamente 12.000 ítems de stock. La mayoría de los
fabricantes depende de sus proveedores para piezas especializadas como por ejemplo
motores eléctricos, cabinas, relees de contacto, guías, puertas metalizas y cerraduras.
Al contrario de esto, elevadores Hércules S.A. tiene la directriz de ser autosuficiente y
producir todas las piezas que utiliza. De esto resulta que la empresa tiene una producción
bastante diversificada, que no es común en su ramo y que da origen a un complejo
sistema de planeamiento y control de la producción.\\
La producción de elevadores no puede seguir un plan general, por que los pedidos
varían considerablemente de acuerdo a las necesidades de los edificios en construcción.
Apenas algunas partes de los elevadores Hércules son Standard y producidas para stock,
como por ejemplo: correderas-guías, guías de puerta, cerradores, motores y conjuntos de
motores generadores, relees de contacto y botones de llamada. El planeamiento de
producción esta dificultado también por el desarrollo tecnológico de la construcción de
diferentes tipos de lugares, dependiendo por eso de condiciones que difícilmente se
pueden prever.\\
El equipo de producción y montaje de elevadores estaba dividido en 4 grupos generales,
de acuerdo con la secuencia a ser seguida en la entrega de partes, conforme al
siguiente esquema:

\begin{itemize}
\item \textbf{Grupo 1}: Modelo soporte para la cabina, guías, correderas, barras, amortiguadores, base, máquina y polea de desvío.
\item \textbf{Grupo 2}: Tablero de comando
\item \textbf{Grupo 3}: Armazón de cabina, contrapesos, paragolpes, plataforma, cabina y cables de acero.
\item \textbf{Grupo 4}: Puertas de lobby, visores, cerraduras, botones de llamada y otros detalles necesarios para que complete el montaje en el edificio.
\end{itemize}

La producción de la fábrica estaba organizada a través de las siguientes secciones:

\begin{enumerate}
\item \textit{Maquinas operativas, tornos, plegadoras, perforadoras, rectificadoras}
\item \textit{Estampado}
\item \textit{Montaje de máquinas}
\item \textit{Montaje de motores}
\item \textit{Montaje de aparatos eléctricos}
\item \textit{Montaje y conexión de cuadros de comando}
\item \textit{Carpintería, fabricación de contrapesos, cabinas y puertas de acero.}
\item \textit{Carpintero, cabinas, puertas y plataformas de madera}
\item \textit{Pintura y galvanoplastia}
\end{enumerate}

En 1970, el planeamiento de producción de elevadores Hércules S.A. era un simple
proceso basado en reportes mensuales de campo del departamento técnico, encargado
del montaje de los elevadores, formado por varios grupos de empleados especializados.
Cada grupo era responsable por el control de una cierta área de la ciudad. El jefe de
grupo visitaba periódicamente a varios clientes de su localidad y estimaba futuras
necesidades. Completaba un formulario de avances del mes donde volcaba los avances
de cada obra indicando el grado de avance de la construcción y estableciendo los
programas de entrega de acuerdo con los cuatro grupos generales del proceso de
producción y montaje ya mencionados. Una vez que el formulario se completaba, le era
entregado al planeador de la producción, un antiguo supervisor que, en 1942, se convirtió
en asistente del departamento de producción a fin de controlar el proceso de
planeamiento de la compañía.\\
A partir de los formularios de avances del mes recibidos por todas las áreas, el
planeador elaboraba el programa de producción para todas las partes a ser producidas
de acuerdo a la secuencia numérica indicada por el departamento de ventas y que
obedecía al orden de entrada de los pedidos de los clientes. El planeador recibía
también las copias de orden de fabricación individual realizadas por el departamento
de ingeniería, conteniendo las especificaciones necesarias para producir cada elevador.
En la época en que la cantidad de elevadores producidos era relativamente baja en
relación con la capacidad de producción de la fabrica, el sistema de planeamiento
descrito, probó ser simple y eficiente y podía ser fácilmente controlado por el planeador
y por los jefes de sección que en conjunto programaban la producción, determinando
cantidades y especificaciones, pidiendo materiales a ser producidos por la fundición, de
oficinas o del pañol.\\
Los reportes mensuales de los grupos de campo eran suficientes para dar al planeador
las informaciones en cuanto a las necesidades futuras de los edificios en construcción y
por lo tanto, esclarecer las prioridades de producción.\\
Entretanto a partir de 1970, el número de construcciones comenzó a aumentar. Los
retrasos en las entregas de elevadores hicieron que los jefes de campo fijasen los
plazos de entrega muy anticipados en sus informes mensuales. Con eso las
informaciones recibidas por el programador fueron perdiendo parte de su valor como
base para la programación. Ocurrió también que ni el planeador ni los jefes de sección
de producción eran avisados cuando un edificio tenía sus obras paralizadas, haciendo
que fuese mantenido el stock de sus correspondientes semielaborados. Este desperdicio
agravaba todavía más la situación de los atrasos provocando graves reclamos por parte
de otros clientes. Teniendo eso en vista, el departamento de ventas comenzó a sugerir
alteraciones en las prioridades distintas a las ordenes de producción, lo que llevo a los
empleados a abandonar los métodos de programación que hasta entonces había sido
establecidos por los jefes de grupo, pasando entonces a trabajar de acuerdo a las
órdenes de ventas del departamento respectivo.\\ \\
\textbf{Decisiones}\\
En vista de la situación, la alta dirección decidió perfeccionar el sistema de planeamiento y control de la fábrica. Contratar una consultora para que analice el caso y revertir la situación de esta compañía.

		\subsubsection{Preguntas}
			\begin{enumerate}
		
			\item \textit{ De acuerdo al enfoque de sistemas, caracterizar a la organización de referencia indicando las entradas, las salidas, la retroalimentación y el ambiente}\\
			Respuesta		
			
			\item \textit{¿Cúal es el problema evidente que enfrenta la empresa}\\
			Respuesta			
			
			\item \textit{¿Cúal es la parte de la estructura que surge como clave?}\\
			Respuesta
			
			\item \textit{¿Cúal es el mecanismo de coordinación preponderante y por qué?}\\
			Respuesta
						
			\item \textit{Dibujar el organigrama actual de la companía}\\
			Respuesta
						
			\item \textit{Mencionar la descripción de uno de los cargos}\\
			Respuesta
						
			\item \textit{Mencionar que tipo de configuración estructural se adapta mejor a la organización de referencia}\\
			Respuesta
						
			\item \textit{Dibujar el organigrama que refleja el cambio propuesto por ustedes}\\
			Respuesta
						
			\item \textit{Razonar como la propuesta hecha puede solucionar los problemas planteados}\\
			Respuesta
			
			\item \textit{Conclusiones}\\
			Respuesta
			
			\end{enumerate}
	\subsection{Caso 2: Los Gringos S.A.}
	\subsubsection{Enunciado}
	\subsubsection{Preguntas}
	
	\begin{enumerate}
		
			\item \textit{ De acuerdo al enfoque de sistemas, caracterizar a la organización de referencia indicando las entradas, las salidas, la retroalimentación y el ambiente}\\
			Respuesta		
			
			\item \textit{¿Cúal es el problema evidente que enfrenta la empresa}\\
			Respuesta			
			
			\item \textit{¿Cúal es la parte de la estructura que surge como clave?}\\
			Respuesta
			
			\item \textit{¿Cúal es el mecanismo de coordinación preponderante y por qué?}\\
			Respuesta
						
			\item \textit{Dibujar el organigrama actual de la companía}\\
			Respuesta
						
			\item \textit{Mencionar la descripción de uno de los cargos}\\
			Respuesta
						
			\item \textit{Mencionar que tipo de configuración estructural se adapta mejor a la organización de referencia}\\
			Respuesta
						
			\item \textit{Dibujar el organigrama que refleja el cambio propuesto por ustedes}\\
			Respuesta
						
			\item \textit{Razonar como la propuesta hecha puede solucionar los problemas planteados}\\
			Respuesta
			
			\item \textit{Conclusiones}\\
			Respuesta
			
			\end{enumerate}
			
	\subsection{Caso 3: La Rapidez}
	\subsubsection{Enunciado}
	\subsubsection{Diagnóstico}
	
	
\section{Conclusiones}
Chamuyo

\begin{comment}
\begin{thebibliography}{99}

\bibitem{INT06} Intel Technology \& Research, ``Hyper-Threading Technology,'' 2006, http://www.intel.com/technology/hyperthread/.

\bibitem{HEN00} J. L. Hennessy and D. A. Patterson, ``Computer Architecture. A Quantitative
Approach,'' 3ra Edición, Morgan Kaufmann Publishers, 2000.

\bibitem{LAR92} J. Larus and T. Ball, ``Rewriting Executable Files to Mesure Program Behavior,'' Tech. Report 1083, Univ. of Wisconsin, 1992.

\end{thebibliography}
\end{comment}
\end{document}
