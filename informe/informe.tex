\documentclass[a4paper,10pt]{article}

\usepackage{graphicx}
\usepackage[utf8]{inputenc}
\usepackage[spanish]{babel}
\usepackage{hyperref}
\usepackage{listings}
\usepackage{verbatim}
\usepackage[top=2cm, bottom=1.5cm, left=2.5cm, right=1cm]{geometry}
\usepackage{pdfpages}
\usepackage{enumitem}

\title{		\textbf{Título}}

\author{	Lucas Simonelli, \textit{Padrón Nro. 93111}                     \\
            \texttt{ lucasp.simonelli@gmail.com }                                              \\[2.5ex]
            Tomás Boccardo, \textit{Padrón Nro. 93637}                     \\
            \texttt{ tomasboccardo@gmail.com}                                              \\[2.5ex]
            Andrés Sanabria, \textit{Padrón Nro. 93403}                     \\
            \texttt{ andresg.sanabria@gmail.com  }                                              \\[2.5ex]
             Agregate master, \textit{Padrón Nro. 9XXXX}                     \\
            \texttt{a@a.com  }                                              \\[2.5ex]
            \normalsize{2do. Cuatrimestre de 2013}                                      \\
            \normalsize{71.12, Estructura de las organizaciones}  \\
            \normalsize{Facultad de Ingeniería, Universidad de Buenos Aires}            \\
       }
\date{}
\begin{document}





\maketitle
\thispagestyle{empty}   % quita el nómero en la primer pógina



\begin{abstract}
Acá va un resumen del trabajo práctico
\end{abstract}

\newpage
\tableofcontents
\newpage
\section{Introducción}
Intro
\section{Empresa relevada}
	\subsection{Preguntas}
		\subsubsection{Preguntas introductorias}
			\begin{enumerate}
				\item ¿Cuál es la razón social de la empresa?
				
				Respuesta
				
				\item ¿En qué lugar se encuentra localizada la empresa?
				
				Respuesta
				
				\item ¿En qué año fue creada la organización?
				
				Respuesta
				
				\item Mencione los acontecimientos más destacados en la evolución de la empresa.
				
				Respuesta
				
				\item ¿Cuál es el rubro al que se dedica la empresa?
				
				Respuesta
				
				\item ¿Cuál es la línea de productos ofrecida por la organización?
				
				Respuesta
				
				\item ¿La empresa cumple algún tipo de certificación?	
				
				Respuesta
				
			
			\end{enumerate}
			
			

		\subsubsection{Producción}
			\begin{enumerate}[resume]
				\item ¿En qué consiste el proceso productivo?
				
				Respuesta
				
				\item ¿Cómo se mide la capacidad de producción?
				
				Respuesta
				
				\item ¿Cuáles son las principales materias primas utilizadas en los procesos de producción?
				
				Respuesta
				
				\item ¿Cuáles son los proveedores más importantes de la empresa?
				
				Respuesta
				
				\item ¿Produce algún insumo necesario para la manufactura del producto final?
				
				Respuesta
				
				\item ¿Hay control de calidad?¿Cómo es? 
								
				Respuesta
				
				\item Maquinaria que se utiliza (tiempo de vida).
				
				Respuesta
				
				\item ¿Poseen un equipo que se encargue del mantenimiento, o es un servicio tercerizado?
				
				Respuesta
				
				\item ¿Realizan un plan de producción? ¿Cómo?
				
				Respuesta
				
				\item ¿Necesita operarios calificados? ¿Realiza algún tipo de capacitación?
				
				Respuesta
				
				\item ¿Cómo se planifica la distribución de los pedidos?
				
				Respuesta
				
				
			
			\end{enumerate}
			
		\subsubsection{Comercial}
		
		
			\begin{enumerate}[resume]
			
				\item ¿Cuáles son los productos más vendidos por la empresa?
			
				Respuesta
				
				\item ¿Qué productos son importados por la organización?
				
				Respuesta
				
				\item ¿Qué productos son exportados por la empresa?
				
				Respuesta
				
				\item ¿Cuáles son los principales destinos de las exportaciones?
				
				Respuesta
				
				\item ¿Qué porcentaje del mercado concentran los productos fabricados por la organización?
				
				Respuesta
				
				\item ¿Qué métodos de publicidad utiliza la empresa?
				
				Respuesta
				
				\item ¿Cuáles son los principales clientes de la empresa?
				
				Respuesta
				
				\item ¿Como se realizan los pedidos?
				
				Respuesta
				
			\end{enumerate}
		\subsubsection{R.R. H.H.}
		
		
			\begin{enumerate}[resume]

			\item ¿Cuántos empleados tiene la organización?
			
			Respuesta
				
			\item ¿Cuántos empleados trabajan en cada área?
			
			Respuesta
				
			\item ¿Qué beneficios posee el personal de la empresa?
			
			Respuesta
			
			\item ¿Cuáles son los principales sectores de la empresa?
			
			Respuesta
			
			\item ¿Tiene la empresa un organigrama propio?(si)
			
			Respuesta
			
			\item ¿Puede la empresa proporcionar dicho organigrama?(Si)
			
			Respuesta
			
			\item ¿Cómo es el proceso de selección del personal? 
			
			Respuesta
			
			\item ¿Se realizan evaluaciones periódicas del desempeño del personal?
			
			Respuesta
			
			\item ¿Hay bonificaciones salariales por buen desempeño?
			
			Respuesta
			
			\item ¿Se capacita a los empleados?
			
			Respuesta
			
			\item ¿Se organizan eventos para fomentar la relación interpersonal entre los empleados?
			
			Respuesta
			
			\item ¿Contratan servicios tercerizados, como por ejemplo: estudio contable, jurídico, controles de calidad, etc.?

			\end{enumerate}
			
			
		\subsubsection{Finanzas}
		
		
			\begin{enumerate}[resume]

			\item ¿Cuál es el nivel de inversión de la empresa?
			
			Respuesta
			
			\item ¿Cómo es el ciclo de pagos?

			Respuesta
			
			\item ¿Cuáles son los plazos promedio para la cobranza de las facturas?
			
			Respuesta
			
			\item ¿Cómo es el circuito de pagos a proveedores?
			
			Respuesta
			
			\item ¿Recibe algún beneficio impositivo por parte del estado?
			Respuesta
			
			\end{enumerate}
\section{Casos de estudio}
	\subsection{Caso 1}
	\subsection{Caso 2}
	\subsection{Caso 3}
	
\section{Conclusiones}
Chamuyo

\begin{comment}
\begin{thebibliography}{99}

\bibitem{INT06} Intel Technology \& Research, ``Hyper-Threading Technology,'' 2006, http://www.intel.com/technology/hyperthread/.

\bibitem{HEN00} J. L. Hennessy and D. A. Patterson, ``Computer Architecture. A Quantitative
Approach,'' 3ra Edición, Morgan Kaufmann Publishers, 2000.

\bibitem{LAR92} J. Larus and T. Ball, ``Rewriting Executable Files to Mesure Program Behavior,'' Tech. Report 1083, Univ. of Wisconsin, 1992.

\end{thebibliography}
\end{comment}
\end{document}
