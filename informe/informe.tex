\documentclass[a4paper,10pt]{article}

\usepackage{graphicx}
\usepackage[utf8]{inputenc}
\usepackage[spanish]{babel}
\usepackage{hyperref}
\usepackage{listings}
\usepackage{verbatim}
\usepackage[top=2cm, bottom=1.5cm, left=2.5cm, right=1cm]{geometry}
\usepackage{pdfpages}
\usepackage{enumitem}

\title{		\textbf{Título}}

\author{	Lucas Simonelli, \textit{Padrón Nro. 93111}                     \\
            \texttt{ lucasp.simonelli@gmail.com }                                              \\[2.5ex]
            Tomás Boccardo, \textit{Padrón Nro. 93637}                     \\
            \texttt{ tomasboccardo@gmail.com}                                              \\[2.5ex]
            Andrés Sanabria, \textit{Padrón Nro. 93403}                     \\
            \texttt{ andresg.sanabria@gmail.com  }                                              \\[2.5ex]
             Agregate master, \textit{Padrón Nro. 9XXXX}                     \\
            \texttt{a@a.com  }                                              \\[2.5ex]
            \normalsize{2do. Cuatrimestre de 2013}                                      \\
            \normalsize{71.12, Estructura de las organizaciones}  \\
            \normalsize{Facultad de Ingeniería, Universidad de Buenos Aires}            \\
       }
\date{}
\begin{document}





\maketitle
\thispagestyle{empty}   % quita el nómero en la primer pógina



\begin{abstract}
Acá va un resumen del trabajo práctico
\end{abstract}

\newpage
\tableofcontents
\newpage
\section{Introducción}
Intro
\section{Empresa relevada}
	\subsection{Preguntas}
		\subsubsection{Preguntas introductorias}
			\begin{enumerate}
				\item \textit{¿Cuál es la razón social de la empresa?}\\
				Tecnología Contra Incendios S.A.
				
				\item \textit{¿En qué lugar se encuentra localizada la empresa?}\\
				La empresa está ubicada en Carapachai, Munro. La dirección es Gobernador Emilio Castro 3365.
				
				\item \textit{¿En qué año fue creada la organización?}\\
				La organización fue creada en 1965.				
				
				\item \textit{Mencione los acontecimientos más destacados en la evolución de la empresa.}\\
				La empresa nace en el año 1965, bajo el nombre de \textbf{TECIN ARGENTINA S.R.L.} Al poco tiempo se transforma en una S.A. y obtiene la representación y distribución, de empresas internacionales de reconocida trayectoria en su especialidad como Rosenbauer, Angus, Reliable, Walter Kidde, Total y Cerberus. En 1982 la empresa construye la primer autobomba argentina tras asociarse con la empresa austriaca Rosenbauer K.G. (hoy Rosenbauer International A.G.).\\
				En la década del 90, sus accionistas deciden formar sociedades independientes para atender unidades de negocios y mercados diferentes, comenzando sus actividades: 
				\begin{itemize}
					\item \textbf{Tecin Instalaciones SA.}, firma especializada en el diseño, instalación y servicio de sistemas contra incendio. Hoy forma parte de Kidde Argentina S.A.

					\item \textbf{Tecin Rosenbauer do Brasil S.A.}, para la fabricación de vehículos contra incendio en el vecino país, y que actualmente por un convenio firmado entre los accionistas locales y Rosenbauer International A.G., pasó a ser propiedad de esta última.

					\item \textbf{Tecnología Contra Incendios S.A.}, es hoy 100\% propiedad de accionistas argentinos y esta dedicada a la fabricación de vehículos y equipos contra incendios y rescate.
				\end{itemize}
				\item \textit{¿Cuál es el rubro al que se dedica la empresa?}\\
				Equipos de seguridad contra incendios (vehículos/equipos/herramientas).
			
				\item \textit{¿Cuál es la línea de productos ofrecida por la organización?}\\	
				La línea de productos ofrecida es de:\\
				\textbf{Vehículos}:
				\begin{itemize}
					\item Contra incendios urbanos, industriales y forestales.
					\item De rescate y manejo de sustancias peligrosas.
				\end{itemize}
				\textbf{Productos/equipos}:
				\begin{itemize}
					\item Motobombas portátiles.
					\item De protección personal y respiratoria.
					\item Para controlar incendios: mangueras, matafuegos, etc.
				\end{itemize}	
												
				\item \textit{¿La empresa cumple algún tipo de certificación?}\\
				Sí, la empresa está certificada bajo las normas IRAM ISO 9001:2008
						
			\end{enumerate}
			
			

		\subsubsection{Producción}
			\begin{enumerate}[resume]
				\item \textit{¿En qué consiste el proceso productivo?}\\
				
				Respuesta
				
				\item \textit{¿Cómo se mide la capacidad de producción?}\\
				
				Respuesta
				
				\item \textit{¿Cuáles son las principales materias primas utilizadas en los procesos de producción?}\\
				
				Respuesta
				
				\item \textit{¿Cuáles son los proveedores más importantes de la empresa?}\\
				
				Respuesta
				
				\item \textit{¿Produce algún insumo necesario para la manufactura del producto final?}\\
				
				Respuesta
				
				\item \textit{¿Hay control de calidad?¿Cómo es?}\\
								
				Respuesta
				
				\item \textit{Maquinaria que se utiliza (tiempo de vida).}\\
				
				Respuesta
				
				\item \textit{¿Poseen un equipo que se encargue del mantenimiento, o es un servicio tercerizado?}\\
				
				Respuesta
				
				\item \textit{¿Realizan un plan de producción? ¿Cómo?}\\
				
				Respuesta
				
				\item \textit{¿Necesita operarios calificados? ¿Realiza algún tipo de capacitación?}\\
				
				Respuesta
				
				\item \textit{¿Cómo se planifica la distribución de los pedidos?}\\
				
				Respuesta
				
				
			
			\end{enumerate}
			
		\subsubsection{Comercial}
		
		
			\begin{enumerate}[resume]
			
				\item \textit{¿Cuáles son los productos más vendidos por la empresa?}\\
			
				Respuesta
				
				\item \textit{¿Qué insumos son importados por la organización?}\\
				
				Respuesta
				
				\item \textit{¿Qué productos son exportados por la empresa?}\\
				
				Respuesta
				
				\item \textit{¿Cuáles son los principales destinos de las exportaciones?}\\
				
				Respuesta
				
				\item \textit{¿Qué porcentaje del mercado concentran los productos fabricados por la organización?}\\
				
				Respuesta
				
				\item \textit{¿Qué métodos de publicidad utiliza la empresa?}\\
				
				Respuesta
				
				\item \textit{¿Cuáles son los principales clientes de la empresa?}\\
				
				Respuesta
				
				\item \textit{¿Como se realizan los pedidos?}\\
				
				Respuesta
				
			\end{enumerate}
		\subsubsection{R.R. H.H.}
		
		
			\begin{enumerate}[resume]

			\item \textit{¿Cuántos empleados tiene la organización?}\\
			La empresa tiene alrededor de 50 empleados.
				
			\item \textit{¿Cuántos empleados trabajan en cada área?}\\
			
			Respuesta
				
			\item \textit{¿Qué beneficios posee el personal de la empresa?}\\
			
			Respuesta
			
			\item \textit{¿Cuáles son los principales sectores de la empresa?}\\
			
			Respuesta
			
			\item \textit{¿Tiene la empresa un organigrama propio?}\\
			Si, la empresa posee organigrama propio.
					
			\item \textit{¿Puede la empresa proporcionar dicho organigrama?}\\
			Si, la empresa puede proporcionar dicho organigrama
			
			\item \textit{¿Cómo es el proceso de selección del personal? }\\
			
			Respuesta
			
			\item \textit{¿Se realizan evaluaciones periódicas del desempeño del personal?}\\
			
			Respuesta
			
			\item \textit{¿Hay bonificaciones salariales por buen desempeño?}\\
			
			Respuesta
			
			\item \textit{¿Se realizan capacitaciones periódicas a los empleados?}\\
			
			Respuesta
			
			\item \textit{¿Se organizan eventos para fomentar la relación interpersonal entre los empleados?}\\
			
			Respuesta
			
			\item \textit{¿Contratan servicios tercerizados, como por ejemplo: estudio contable, jurídico, controles de calidad, etc.?}\\

			\end{enumerate}
			
			
		\subsubsection{Finanzas}
		
		
			\begin{enumerate}[resume]

			\item \textit{¿Cuál es el nivel de inversión de la empresa?}\\
			
			Respuesta
			
			\item \textit{¿Cómo es el ciclo de pagos?}\\

			Respuesta
			
			\item \textit{¿Cuáles son los plazos promedio para la cobranza de las facturas?}\\
			
			Respuesta
			
			\item \textit{¿Cómo es el circuito de pagos a proveedores?}\\
			
			Respuesta
			
			\item \textit{¿Recibe algún beneficio impositivo por parte del estado?}\\
			Respuesta
			
			\end{enumerate}
		
		\subsubsection{Control de Calidad}
		
		
			\begin{enumerate}[resume]

			\item \textit{¿Cuáles es la política de calidad de la organización?}\\
			La política de calidad de la empresa presenta los siguientes objetivos:
			\begin{itemize}
				\item El cumplir con los compromisos contraídos con nuestros clientes y superar sus expectativas, constituyen una obligación para todo nuestro personal.
				\item Asumir como  indispensable el cumplimiento  de los requerimientos de nuestro Sistema de 
Gestión  de la Calidad y comprometernos a su continua mejora.
				\item Nos comprometemos a evaluar, motivar y capacitar a nuestros recursos humanos, en forma 
permanente. 
				\item Mantener un contacto productivo con nuestros Proveedores, para mejorar las prestaciones 
y productos ofrecidos. 
				\item Difundir a nuestro personal los objetivos comprometidos en la presente política. 
			\end{itemize}
	
			\item \textit{¿Cómo realizan los controles de calidad sobre la producción?}\\
			
			Respuesta
			
			\item \textit{¿Realizan auditorías internas? ¿Con qué frecuencia?}\\

			Respuesta
			
			\item \textit{¿Realizan auditorías externas? ¿Con qué frecuencia?}\\
			Respuesta
			
			\end{enumerate}	
			
\section{Casos de estudio}
	\subsection{Caso 1}
	\subsection{Caso 2}
	\subsection{Caso 3}
	
\section{Conclusiones}
Chamuyo

\begin{comment}
\begin{thebibliography}{99}

\bibitem{INT06} Intel Technology \& Research, ``Hyper-Threading Technology,'' 2006, http://www.intel.com/technology/hyperthread/.

\bibitem{HEN00} J. L. Hennessy and D. A. Patterson, ``Computer Architecture. A Quantitative
Approach,'' 3ra Edición, Morgan Kaufmann Publishers, 2000.

\bibitem{LAR92} J. Larus and T. Ball, ``Rewriting Executable Files to Mesure Program Behavior,'' Tech. Report 1083, Univ. of Wisconsin, 1992.

\end{thebibliography}
\end{comment}
\end{document}
